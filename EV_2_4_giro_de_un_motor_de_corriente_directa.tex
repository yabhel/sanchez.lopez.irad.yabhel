\documentclass[10pt,a4paper]{article}
\usepackage[utf8]{inputenc}
\usepackage{amsmath}
\usepackage{amsfonts}
\usepackage{amssymb}
\usepackage{graphicx}
\usepackage[left=2cm,right=2cm,top=2cm,bottom=2cm]{geometry}
\author{Irad Yabhel Sanchez Lopez}
\title{MOTORES DE CORRIENTE DIRECTA }
\begin{document}

\title{•UNIVERSIDAD POLITECNICA DE LA ZONA METROPOLITANA DE GUADALAJARA}


\chapter{MOTORES DE CORRIENTE DIRECTA}


\begin{flushleft}
El motor de corriente continua, denominado también motor de corriente directa, motor CC o motor DC, es una máquina que convierte energía eléctrica en mecánica, provocando un movimiento rotatorio, gracias a la acción de un campo magnético. Un motor de corriente continua se compone, principalmente, de dos partes:.
\n

La corriente contínua presenta grandes ventajas, entre las cuales está su capacidad para ser almacenada de una forma relativamente sencilla. Esto, junto a una serie de características peculiares de los motores de corriente contínua, y de aplicaciones de procesos electrolíticos, tracción eléctrica, entre otros, hacen que existen diversas instalaciones que trabajan basándose en la corriente continua.

Los generadores de corriente contínua son las mismas máquinas que transforman la energía mecánica en eléctrica. No existe diferencia real entre un generador y un motor, a excepción del sentido de flujo de potencia. Los generadores se clasifican de acuerdo con la forma en que se provee el flujo de campo, y éstos son de excitación independiente, derivación, serie, excitación compuesta acumulativa y compuesta diferencial, y además difieren de sus características terminales (voltaje, corriente) y por lo tanto en el tipo de utilización.

Durante el desarrollo del presente informe, el enfoque se hará en relación con el principio de funcionamiento de las distintas versiones de máquinas eléctricas de corrientes continua que existen, dado el amplio campo para las cuales son utilizadas. El entendimiento de tales máquinas, permiten al ingeniero una eficaz elección además de la posibilidad de evitar situaciones en las que se produzcan accidentes a causa del uso u operación inadecuada de los equipos que trabajan con este tipo de energía. Los conocimientos previos de teoría básica de circuitos eléctricos, serán de gran ayuda para comprender las funciones de cada uno de los componentes de las máquinas de corriente contínua.

\n

Las máquinas de corriente contínua son generadores que convierten energía mecánica en energía eléctrica de corriente continua, y motores que convierten energía eléctrica de corriente continua en energía mecánica. La mayoría las máquinas de corriente contínua son semejantes a las máquinas de corriente alterna ya que en su interior tienen corrientes y voltajes de corriente alterna. Las máquinas de corriente contínua tienen corriente contínua sólo en su circuito exterior debido a la existencia de un mecanismo que convierte los voltajes internos de corriente alterna en voltajes corriente continua en los terminales. Este mecanismo se llama colector, y por ello las máquinas de corriente continua se conocen también como máquinas con colector.

\n

Las partes de un motor de corriente directa son:
\n
Culata
Núcleo polar
Pieza polar
Núcleo de polo auxiliar
Pieza polar de polo auxiliar
Inducido
Arrollado del inducido
Arrollado de excitación
Arrollado de conmutación
Colector
Escobillas positivas
Escobillas negativas
\n
\includegraphics[scale=1]{11.jpg} 
\includegraphics[scale=1]{12.jpg} 

\n

La máquina de corriente contínua consta básicamente de las partes siguientes:

Inductor: Es la parte de la máquina destinada a producir un campo magnético, necesario para que se produzcan corrientes inducidas, que se desarrollan en el inducido.
El inductor consta de las partes siguientes:

Pieza polar: Es la parte del circuito magnético situada entre la culata y el entrehierro, incluyendo el núcleo y la expansión polar.
Núcleo: Es la parte del circuito magnético rodeada por el devanado inductor.
Devanado inductor: es el conjunto de espiras destinado a producir el flujo magnético, al ser recorrido por la corriente eléctrica.
Expansión polar: es la parte de la pieza polar próxima al inducido y que bordea al entrehierro.
Polo auxiliar o de conmutación: Es un polo magnético suplementario, provisto o no, de devanados y destinado a mejorar la conmutación. Suelen emplearse en las máquinas de mediana y gran potencia.
Culata: Es una pieza de sustancia ferromagnética, no rodeada por devanados, y destinada a unir los polos de la máquina.
Inducido: Es la parte giratoria de la máquina, también llamado rotor.

\end{document}
\end{flushleft}
\documentclass[11pt,a4paper]{article}
\usepackage{graphicx}
\usepackage{amsmath}
\usepackage{amssymb}
\usepackage{mathrsfs}
\usepackage{cancel}

\begin{document}
\begin{center}
\textbf{REPORTE DE PROYECTO}\\
\textbf{SISTEMA DE RIEGO}
\end{center}

\begin{center}
Cedano Gutierrez Nancy Noemi\\
Gomez Islas Maria de Lourdes\\
Sanchez Lopez Irad Yabhel\\
Banda Macias Francisco Javier\\
\textbf{20-sep-2019}\\
\textbf{Universidad Politecnica de La Zona Metropolitana de Guadalajara}
\end{center}

\section{Introduccion}
El tiempo es algo que hasta ahora es imposible de detener los años pasan y con ellos va avanzando nuetsra tecnologia.\\
En la Universidad Politecnica de la Zona Metropolitana de Guadalajara, esta comenzando a plantar arboles en sus alrededores, estudiantes de diferentes grados siembran arboles simplemente por calificacion y luego los olvidan. Esto ah sido un problema ya que no se cuidan ni se riegan.\\
Como equipo, se creo un sistema de riego automatico que pueden cubrir parcialmente toda la universidad hatsa todo nuestra ecosistema.\\
Para comenzar con este proyecto se esta comenzando con los arboles recien plantados para asi ir mejorando dia con dia hasta llegar a cubrir ciertas partes de nuestro planeta.\\
El sistema de riego (\emph{4live}) consiste en controlar desde una aplicacion de telefono inteligente el sistema de compuertas de agua donde la energia adquirida para dicho acceso de compuerta de panel solar donde absorvera los rayos emitidos por el sol, de esta manera es probable tener mas arboles fuertes y verdes ya que no les hara falta agua, gracias a este sistema (\emph{4live}).

\section{Objetivo}
Lograr una tarea eficiente, como regar las areas verdes de nuestra universidad o de todo el ecosistema deseando de una manera mas tecnologica y segura.

\section{Materiales de la estructura:}
\begin{enumerate}
\item Caja de proteccion
\item Contenedor de agua
\item Panel solar
\item Tubo Upvc
\item Arduino
\item Codos de PVC
\item Sistema de paneles solares
\item Controladores inalambricos
\item Cemento de ladrillos
\item 16 compuertas
\item Programacion del sistema
\end{enumerate}

\begin{center}
\subsection{MATERIAS RELACIONADAS ACTUALMENTE PARA EL PROYECTO:}
\end{center}

\subsubsection{CONTROLADORES LOGICOS PROGRAMABLES}
como el sistema lleva controladores inalambricos, por el uso de arduino que indicara cuando las 16 compuertas se cierren o abran utilizando el \textbf{relevador BUFFER}.
\subsubsection{ESTRUCTURA Y PROPIEDADES DE LOS MATERIALES}
Resistencia del material tanto como la temperatura del panel solar.
\subsubsection{PROGRAMACION DE PERIFERICOS}
El diseño de la aplicacion añadiendo datos de entrada, un proceso y datos de salida. (\emph{en este caso utilizando arduino}). 
\subsubsection{SISTEMAS ELECTRONICOS DE INTERFAZ}
Armado de circuito con relevadores.

\section{gastos}
\begin{itemize}
\item  Caja  de  proteccion = \textbf{90}
\item  Contenedor de agua = \textbf{800 }
\item  Panel solar = \textbf{3,000}
\item  Turvy Upvc = \textbf{100} 
\item  Arduino = \textbf{150} 
\item  Codos de PVC = \textbf{20} 
\item  Sistema de paneles solares = \textbf{/} 
\item  Controladores inalambricos = \textbf{500} 
\item  Cemento de ladrillos = \textbf{ / }
\item  16 compuertas = \textbf{200 c/u }
\item  Programacion del sistema = \textbf{/} 
\end{itemize}

\subsection{Sumatoria de los costos }
$ \textbf{5,500} $

\subsection{Respartimiento de los gastos}
\begin{itemize}
\item \emph{Cedano Gutierrez Nancy Noemi} = \textbf{1375}
\item \emph{Gomez Islas Maria de Lourdes} = \textbf{1375}
\item \emph{Sanchez Lopez Irad Yabhel} = \textbf{1375}
\item \emph{Banda Macias Francisco Javier} = \textbf{1375}
\end{itemize}


\end{document}
\documentclass[10pt,a4paper]{article}
\usepackage[utf8]{inputenc}
\usepackage{amsmath}
\usepackage{amsfonts}
\usepackage{amssymb}
\usepackage{graphicx}
\usepackage[left=2cm,right=2cm,top=2cm,bottom=2cm]{geometry}
\author{Irad Yabhel Sanchez Lopez}
\title{Amplificadores Clase A }
\begin{document}

UNIVERSIDAD POLITECNICA DE LA ZONA METROPOLITANA DE GUADALAJARA


Amplificadores Clase A

Amplificador clase A. Son aquellos amplificador cuyas etapas de potencia consumen corrientes altas y continuas de su fuente de alimentación.
Características
Esta amplificación presenta el inconveniente de generar una fuerte y constante emisión de calor. No obstante, los transistores de salida están siempre a una temperatura fija y sin alteraciones.
En general, se afirma que esta clase de amplificación es frecuente en circuitos de audio y en los equipos domésticos de gama alta, ya que proporcionan una calidad de sonido potente y de muy buena calidad.
Los amplificadores de clase A a menudo consisten en un transistor de salida conectado al positivo de la fuente de alimentación y un transistor de corriente constante conectado de la salida al negativo de la fuente de alimentación.
La señal del transistor de salida modula tanto el voltaje como la corriente de salida. Cuando no hay señal de entrada, la corriente de polarización constante fluye directamente del positivo de la fuente de alimentación al negativo, resultando que no hay corriente de salida, se gasta mucha corriente. Algunos amplificadores de clase A más sofisticados tienen dos transistores de salida en configuración push-pull.
Ventaja
La clase A se refiere a una etapa de salida con una corriente de polarización mayor que la máxima corriente de salida que dan, de tal forma que los transistores de salida siempre están consumiendo corriente. La gran ventaja de la clase A es que es casi lineal, y en consecuencia la distorsión es menor.
Desventaja
La gran desventaja de la clase A es que es poco eficiente, se requiere un amplificador de clase A muy grande para dar 50 W, y ese amplificador usa mucha corriente y se pone a muy alta temperatura.
¿Cuáles son los amplificadores clase A?
Un amplificador de conmutación o amplificador clase D es un amplificador electrónico el cual, en contraste con los amplificadores clase AB cuyos transistores de potencia operan en modo lineal (región activa), usa el modo conmutado de los transistores (corte y saturación) para regular la entrega de potencia.


Clasificación de los amplificadores
1. Clasificación de los amplificadores
La primera clasificación que podemos hacer con los amplificadores viene determinada por las frecuencias con las que van a trabajar. Si las frecuencias están comprendidas dentro de la banda audible los amplificadores reciben el nombre de amplificadores de audio frecuencia o amplificadores de Baja frecuencia. (amplificadores A.F. o amplificadores B.F., respectivamente).
Dentro de las dos gamas de amplificadores vistas, también, podemos hacer una clasificación atendiendo a su forma de trabajo:
a) Amplificadores de tensión: son los que su principal misión es suministrar una tensión mayor en su salida que en su entrada
b) Amplificadores de potencia: aquellos que, aparte de suministrar una mayor tensión, suministran también un mayor corriente (amplificación de tensión y amplificación de corriente y, por ende, amplificación de potencia).
A. Amplificadores de clase A: un amplificador de potencia funciona en clase A cuando la tensión de polarización y la amplitud máxima de la señal de entrada poseen valores tales que hacen que la corriente de salida circule durante todo el período de la señal de entrada.
B. Amplificadores de clase B: un amplificador de potencia funciona en clase B cuando la tensión de polarización y la amplitud máxima de la señal de entrada poseen valores tales que hacen que la corriente de salida circule durante un semiperíodo de la señal de entrada.
C. Amplificadores de clase AB: son, por así decirlo, una mezcla de los dos anteriores, un amplificador de potencia funciona en clase AB cuando la tensión de polarización y la amplitud máxima de la señal de entrada poseen valores tales que hacen que la corriente de salida circule durante menos de un período y más de un semiperíodo de la señal de entrada.
D. Amplificadores de clase C: un amplificador de potencia funciona en clase C cuando la tensión de polarización y la amplitud máxima de la señal de entrada poseen valores tales que hacen que la corriente de salida circule durante menos de un semiperíodo de la señal de entrada.

En resumen existen muchos tipos de clases de amplificadores, los de tipo A que son de los mejores en su clasificacion esto es porque la fuerza de la corriente que entra y sale es mayor esto se debe al transistor que entre mas grande es mayor la corriente y la tension de salida quiza la ganancia es igual pero el ultimo transistor de salida.

Lo primero que se tiene que contemplar es el trasnsitor es el corazon de los amplificadores y lo ideal es buscar la polarizacion.

Es importante tambien buscar el punto Q conocerlo y buscar polarizarlo para que no exista una saturacion lo ideal es buscar un punto medio exacto.


\end{document}
\documentclass[11pt,a4paper]{article}
\usepackage{graphicx}
\usepackage{amsmath}
\usepackage{amssymb}
\usepackage{mathrsfs}
\usepackage{cancel}

\begin{document}
\begin{center}
\textbf{Practica3}\\ 
EV_2_1_Diseño_de_puente_H
\end{center}

\begin{center}
Sanchez Lopez Irad Yabhel\\
29-OCT-2019\\
Universidad Politecnica de La Zona Metropolitana de Guadalajara
\end{center}


\section{Objetivo de la Practica}

enerar un circuito realizando el puente H 

¿Qué es un puente H?
El puente H  es un circuito electrónico que permite a un motor eléctrico DC girar en ambos sentidos, avanzar y retrocerder.

Los puentes H ya vienen hechos en algunos circuitos integrados, pero también se pueden construir a partir de componentes eléctricos y/o electronicos.

Un puente H se construye con 4 interruptores (mécanicos o mediante transistores). Cuandos los interruptores S1 y S4 están cerrados ( S2 y S3 abiertos ) se aplica una tensión haciendo girar el motor en un sentido. Abriendo los interruptores S1 y S4 ( cerrando S2 y S3 ), el voltaje se invierte, permitiendo el giro en sentido inverso del motor.

Un puente H no solo se usa para invertir el giro de un motor, también se puede usar para frenarlo de manera brusca, al hacer un corto entre los bornes del motor, o incluso puede usarse para permitir que el motor frene bajo su propia inercia, cuando desconectamos el motor de la fuente que lo alimenta.

básicamente se puede hacer esto tomando en cuenta la siguiente tabla.

Material requerido para le armado del proto.
2 transistores 2N2222 que son los que conmutaran las salidas.
4 transistores TIP31 los cuales actuaran como interruptores.
4 Diodos rectificadores para crear un puente de diodos a manera que limpien la señal analógica y se pueda usar el motor de corriente directa.

\includegraphics[scale=1]{ph12.png} 


\end{document}
\documentclass[10pt,a4paper]{article}
\usepackage[utf8]{inputenc}
\usepackage{amsmath}
\usepackage{amsfonts}
\usepackage{amssymb}
\usepackage{graphicx}
\usepackage[left=2cm,right=2cm,top=2cm,bottom=2cm]{geometry}
\author{khksvc}
\title{JbsdkbvjlSBV}
\begin{document}

UNIVERSIDAD POLITECNICA DE LA ZONA METROPOLITANA DE GUADALAJARA
\\
\\


CIRCUITOS DE RECTIFICACION NO CONTROLADOS
\\
\\

Objetivo de la practica
\\
Conocer las caracteristicas de y funcionamiento de cada rectificador
\\
Procedimiento con un rectificador de Media Onda
\\
Nos da a conocer algunos elementos de la rectificacion no controlada
\\
\includegraphics[scale=1]{11.png} 
\\

la imagen mustra la simulacion realizada en Orcad  se muestra como la onda de tension rectificada se aplica a la carga
\\
Con esta onda de salida no se anula hasta que no lo hace la corriente de carga, lo que significa que el diodo rectificador permanece polarizado en directo, incluso durante una porcion del semi periodo negativo de la tension de entrada.\\
Esto es debido a la inductancia de salida se opone a variaciones bruscas de corriente y asi crea una sobretencion necesaria para mantener al diodo en conduccion hasta que la corriente sea nula.\\
\\
\includegraphics[scale=1]{12.png} 
\\
\includegraphics[scale=1]{14.png} 
\\
\includegraphics[scale=1]{13.png} 
\\
Algunos ejemplos de las ondas se muestran acontinuacion se ilustran en PSpice asi mismo el rectificador monofasico el cual consta de un filtro constituido por los elementos Cf y Lf destinado atenuar el rizado de la tension de salida.\\
En este circuito podremos estudiar las principales formas de ondas que caracterizan su funcionamiento y determinan la eleccion de los diodos empeorando el factor de potencia.
\\
\includegraphics[scale=1]{15.png} 
\\
\includegraphics[scale=1]{16.png} 
\\
Una vez que se logro identificar la simulacion y las ondas en PSpise y asi mismo la tension y la corrientee de un rectificador procedemos a realizar el uso de cuantificar los resultados con los precedentes obtenidos y determinar el factor de potencia global del rectificador, utilizando las herramientas del fichero de PSpice
\\
\\
Conclusion \\
\\
Me es aun complicado entender muchas de estas nuevas etapas software y sistemas esperando que en lo pronto poder asimilar esta informacion ponerla en practica y adaptarme rapido al entorno por partye de la practica solo segui las instrucciones del manual de practica la cual no fue del todo claro orcad una plataforma la cual da muchas variaciones kicad otra donde indica incluso marcas y series de los componentes latex aun sigo sin entender la logica y "facilidad" que se indica lo mismo con github en fin creo que es cuestion de adaptacion a un modelo acelerado.



\end{document}
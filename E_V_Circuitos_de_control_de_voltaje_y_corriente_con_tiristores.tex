\documentclass[11pt,a4paper]{article}
\usepackage{graphicx}
\usepackage{amsmath}
\usepackage{amssymb}
\usepackage{mathrsfs}
\usepackage{cancel}

\begin{document}
\begin{center}
\textbf{Practica3}\\ 
EV 1_3 Circuitos_de_control_de_voltaje_y_corriente_con_tiristores
\end{center}

\begin{center}
Sanchez Lopez Irad Yabhel\\
29-OCT-2019\\
Universidad Politecnica de La Zona Metropolitana de Guadalajara
\end{center}


\section{Objetivo de la Practica}

realizar un circuito mediante tiristores para encencer un foco el cual sea regulada su intenciadad
\includegraphics[scale=1]{11.png} 
Tiristores.
El término “tiristor”, contracción de TIRatrón y de transISTOR, designa a
una familia de semiconductores de potencia cuyas características originales eran similares a
las de los tubos tiratrones.
Entre la amplia familia de los tiristores, distinguiremos:
· Los tiristores propiamente dichos, conocidos con el nombre de SCR (Silicon Controlled Rectifiers), son elementos unidireccionales con tres salidas
(ánodo, cátodo y puerta) bloqueados en el tercer cuadrante. Pertenecen
igualmente a esta familia los tiristores asimétricos o ASCR (Asymmetrical
Silicon Controlled Rectifiers) y los tiristores RCT (Reverse Controlled
Thyristors), que incluyen en su silicio o su cápsula un diodo en antiparalelo.
· Los triacs, elementos bidireccionales que derivan de los tiristores. Su
nombre proviene de la contracción TRiode AC switch.
· Los GTO (Gate Turn-OffThyristor) o tiristores de conmutación por puerta.
· Los IGBT.
· Los tiristores fotosensibles o fototiristores. El término inglés utilizado para
ellos es LASCR (Light Activated Silicon Controlled Rectifiers).
\includegraphics[scale=1]{13.jpg} 
Los tiristores son una familia de dispositivos semiconductores de cuatro capas (pnpn), que se utilizan para controlar grandes cantidades de corriente mediante circuitos electrónicos de bajo consumo de potencia.

La palabra tiristor, procedente del griego, significa puerta. El nombre es fiel reflejo de la función que efectúa este componente: una puerta que permite o impide el paso de la corriente a través de ella. Así como los transistores pueden operar en cualquier punto entre corte y saturación, los tiristores en cambio sólo conmutan entre dos estados: corte y conducción.

Poniendo en practica lo aprendido se procede a la conexion del circuito en el proto generando asi que el foco encienda y regule su intencidad.
\includegraphics[scale=1]{12.jpg} 
Conclusion:

Una vez que el foco enciende y se regula la intensidad podemos identificar las partes y la composicion dle armado del circuito


\end{document}
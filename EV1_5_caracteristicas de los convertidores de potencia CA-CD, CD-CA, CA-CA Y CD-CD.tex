\documentclass[10pt,a4paper]{article}
\usepackage[utf8]{inputenc}
\usepackage{amsmath}
\usepackage{amsfonts}
\usepackage{amssymb}
\usepackage{graphicx}
\usepackage[left=2cm,right=2cm,top=2cm,bottom=2cm]{geometry}
\author{khksvc}
\title{JbsdkbvjlSBV}
\begin{document}

Que son los convertidores


CONVERTIDORES CD-CD


Los convertidores CD-CD son una de las herramientas dentro de la electrónica de potencia con las que podemos alcanzar altas ganancias de voltaje.

Sus aplicaciones son en fuentes de poder de computadoras, sistemas de paneles solares, cargar dispositivos a partir de suministros energético de un automóvil


CONVERTIDOR DE CD-CA


Estos convertidores de CD-CA también conocidos como inversores es un sistema para converit la corriente continua a un voltaje simetrico de corriente alterna
Algunas de sus aplicaciones son en drivers para motores de corriente alterna, fuentes de alimentación interrumpidas, generación de fotovolcaica y generar corriente alterna de paneles solares


CONVERTIDOR DE CA-CA


Los convertidores de CA-CA están diseñados para controlar el flujo de potencia de corriente alterna o también se le puede llamar controladores de voltaje de CA
Sus aplicaciones en el control de velocidad de motores, hornos industriales, control de iluminación, control de reactivos

Por otra parte los inversores en puente completos estan formados por 4 interruptores de potencia totalmente controlados, regularmente transistores MOSFETS o IBGTS La tensión de salida Vc puede ser +Vcc -Vcc, ó 0 dependiendo del estado de los interruptores


Medio puente

La tensión de salida es una onda cuadrada de amplitud VE/2. su onda de salida es cuadrada con un alto contenido armonico, ademas que su amplitud de salida no es controlable sino variable


push pull (literalmente empujar-halar)

funciona de manera que el tranformador de su circuito se magnetiza y desmagnetiza en un periodo de trabajo. Esta compuesto por una especie de inversor que convierte la tansión continua en alterna utilizando dos transistores y un rectificador de onda completa y un filtro paso bajo.
Las salidas push-pull están presentes en circuitos lógicos digitales TTL y CMOS y en algunos tipos de amplificadores


trifasico 

Es un sistema de producción, distribución y consumo de energía eléctrica formado por tres corrientes alternas monofásicas de igual frecuencia y amplitud, que presentan una diferencia de fase entre ellas de 120 grados eléctricos, y están dadas en un orden determinado.

Los inversores trifásicos se emplean en aplicaciones de baja, media y alta potencia con tensiones de salida de baja y media tensión.
La característica trifásica los hace adecuados para aplicaciones de control de velocidad en motores de inducción de corriente alterna.


Formas de onda cuadrada

Se conoce por onda cuadrada a la onda de corriente alterna (CA) que alterna su valor entre dos valores extremos sin pasar por los valores intermedios (al contrario de lo que sucede con la onda senoidal y la onda triangular, etc.)El dispositivo de conmutación que cambia la dirección de la corriente debe actuar con rapidez A medida que la corriente pasa a través de la cara primaria del transformador, la polaridad cambia 100 veces cada segundo  , en una frecuencia de 50 ciclos completos por segundo. 

La onda cuadrada es una onda estacionaria no-sinusoidal. Lo característico de este tipo de onda es que su serie armónica solo tiene armónicos impares, es decir que la amplitud de los armónicos pares es igual a 0. Esto ocasiona que su forma se asemeje a la de medios cuadrados ya que se conmuta entre los valores máximos y los mínimos con velocidad infinita.


Forma de onda semi cuadrada

La conmutación de onda semi cuadrada trata de eliminar el inconveniente que presenta la conmutación de onda cuadrada (no permite regular la magnitud de la tensión de salida) manteniendo su principal ventaja (cada interruptor activado la mitad del período).

Es decir el ángulo eléctrico en que los interruptores de un mismo nivel se encuentran activados o desactivados al mismo tiempo en el ángulo de solapamiento.  
Controlando dicho ángulo se controla la magnitud de la tensión de salida para cualquier armónico. 
 Por tanto, si el ángulo de solapamiento a igual a 0 grados el inversor de onda semi cuadrada funcionará como uno de conmutacion de onda cuadrada.


Forma de onda modulada

se tienen diversos tipos de onda modulada:

Modulación en doble banda lateral (DSB)
Modulación de amplitud (AM)
Modulación de fase (PM)
Modulación de frecuencia (FM)
Modulación banda lateral única (SSB, ó BLU)
Modulación de banda lateral vestigial (VSB, VSB-AM, ó BLV)

El proceso de modulación se basa en comparar la onda modulante con la onda portadora. La salida del inversor se fija en su valor positivo cuando la amplitud de la sinusoide es superior a la amplitud de la triangular,  y en su valor negativo en el caso contrario.

Conclusion 

todas estas ondas las vemos a diario en el dia a dia, en cada motor que usamos es interesante la forma en que se aplican para eficientar cada una de las actividades o motores que usamos con una proceso para trasnformar la energia y poder cumplir la demanda y generar un ahorro de energia con esto se evitan altos costos ayuda al medio enytre tantos mas beneficios.

los convertidores de frecuencia de corriente alterna o tambien conocidos como convertidores de velocidad variable o de frecuencia variable, el cual se encuentra en medio entre la alimentacion y el motor el cual va a regular toda la energia y la manera en que va a salir el tipo de onda o una fuente de energia limpia y esta llega a un inversor que transforma la corriente continua en alterna de salida al motor esto permitiendo que el convertidor ajuste la frecuencia y la tension en funcion de los requisitos del proceso.

\end{document}
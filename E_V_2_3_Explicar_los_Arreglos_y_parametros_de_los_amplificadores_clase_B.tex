\documentclass[10pt,a4paper]{article}
\usepackage[utf8]{inputenc}
\usepackage{amsmath}
\usepackage{amsfonts}
\usepackage{amssymb}
\usepackage{graphicx}
\usepackage[left=2cm,right=2cm,top=2cm,bottom=2cm]{geometry}
\author{Irad Yabhel Sanchez Lopez}
\title{Amplificadores Clase A }
\begin{document}

UNIVERSIDAD POLITECNICA DE LA ZONA METROPOLITANA DE GUADALAJARA


Amplificadores Clase B

Se les denomina amplificador clase B, cuando el voltaje de polarización y la máxima amplitud de la señal entrante poseen valores que hacen que la corriente de salida circule durante el semiciclo de la señal de entrada. La característica principal de este tipo de amplificadores es el alto factor de amplificación.
Los amplificadores de clase B se caracterizan por tener intensidad casi nula a través de sus transistores cuando no hay señal en la entrada del circuito, por lo que en reposo el consumo es casi nulo.
Sus características son:
Cuando el voltaje de polarización y la máxima amplitud de la señal entrante poseen valores que hacen que la corriente de salida circule durante el semiciclo de la señal de entrada.
La característica principal de este tipo de amplificadores es el alto factor de amplificación.

Siempre son dos transistores (push bull) generalemnete cuando uno esta en activo el otro esta en off y viceversa y la misma funcion el amplificar la señal ya sea negativa o negativa por igual.

Asi mismo se maneja un estandar de marca de 0.6 y -0.6 V pasando esta medida amplifica si no llega a esa medida no arranca ya sea ni uno u otro.

Existe tambien la distorcion de cruce los tienen todoss los tipo B, 


 
Ventajas:
•	Posee bajo consumo en reposo.
•	Aprovecha al máximo la Corriente entregada por la fuente.
•	Intensidad casi nula cuando está en reposo.
Desventajas:
•	Producen armónicos, y es mayor cuando no tienen los transistores de salida con las mismas características técnicas, debido a esto se les suele polarizar de forma que se les introduce una pequeña polarización directa. Con esto se consigue desplazar las curvas y se disminuye dicha distorsión.
Sus aplicaciones
•	Sistemas telefónicos,
•	Transmisores de seguridad portátiles
•	Sistemas de aviso, aunque no en audio.



\end{document}